\documentclass[12pt]{article}
\usepackage{blindtext}
\usepackage[T1]{fontenc}
\usepackage[utf8]{inputenc}
\usepackage{amsmath,amsthm,amssymb,amsfonts}

\usepackage{setspace}
\renewcommand{\baselinestretch}{1.5}



\begin{document}
\title{LaTeX sample file 1}
\author{LLCCMM}
\date{\today}

\maketitle

\section{Introduction}

This is the first section.

An \textbf{economy} is an area of the production, distribution and trade, as well as consumption of goods and services by different agents. Understood in its broadest sense, 'The economy is defined as a social domain that emphasize the practices, discourses, and material expressions associated with the production, use, and management of resources'. A given economy is the result of a set of processes that involves its culture, values, education, technological evolution, history, social organization, political structure and legal systems, as well as its geography, natural resource endowment, and ecology, as main factors. These factors give context, content, and set the conditions and parameters in which an economy functions. In other words, the economic domain is a social domain of human practices and transactions. It does not stand alone.

\section{Second Section}

{\Large Economic agents can be individuals, businesses, organizations, or governments. Economic transactions occur when two groups or parties agree to the value or price of the transacted good or service, commonly expressed in a certain currency. However, monetary transactions only account for a small part of the economic domain.}

\subsection{A Subsection}

 Economic activity is spurred by production which uses natural resources, labor and capital. It has changed over time due to technology (automation, accelerator of process, reduction of cost functions), innovation (new products, services, processes, expanding markets, diversification of markets, niche markets, increases revenue functions) such as, that which produces intellectual property and changes in industrial relations (most notably child labor being replaced in some parts of the world with universal access to education).

\begin{equation}
E = MC^{2}
\end{equation}

\begin{equation*}
E = MC^{2}
\end{equation*}


\end{document}