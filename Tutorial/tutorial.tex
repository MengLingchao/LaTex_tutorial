\documentclass[11pt]{beamer}
\usepackage[utf8]{inputenc}
\usepackage[T1]{fontenc}
\usepackage{lmodern}
\usepackage{amsmath}
\usepackage{amsfonts}
\usepackage{amssymb}
\usepackage{graphicx}
\usetheme{cambridgeUS}
\usepackage[longnamesfirst]{natbib} 
\usepackage{caption}
\DeclareCaptionFormat{citation}{%
	\ifx\captioncitation\relax\relax\else
	\captioncitation\par
	\fi
	#1#2#3\par}
\newcommand*\setcaptioncitation[1]{\def\captioncitation{\textit{Source:}~#1}}
\let\captioncitation\relax
\captionsetup{format=citation,justification=centering}
\usepackage {subcaption}
\usepackage{CJKutf8}


\begin{document}
	
	\begin{CJK*}{UTF8}{gkai}
		
	\author[LC Meng]{Lingchao Meng}
	\title[LaTeX Tutorial]{\LaTeX\ Tutorial}
	%\subtitle{}
	%\logo{}
	\institute[SE, PKU]{School of Economics\\Peking University}
	%\date{}
	%\subject{}
	%\setbeamercovered{transparent}
	%\setbeamertemplate{navigation symbols}{}
\begin{frame}[plain]
    \maketitle
\end{frame}

\begin{frame}
	This is  just a beginners guide to writing documents in \LaTeX\ without prior knowledge of \LaTeX. This slide is designed for the \LaTeX\ workshop at School of Economics, Peking University.
\end{frame}

\begin{frame}
	This file and some other materials can be download from my GitHub repository: \underline{https://github.com/MengLingchao/LaTex\_tutorial}. \\Please feel free to download and use it.
\end{frame}

\begin{frame}{Outline}
	\tableofcontents
\end{frame}

\section{Introduction}
\begin{frame}
	\sectionpage
\end{frame}

\begin{frame}{Definition}
\LaTeX	Working definitions of capital reallocation are based on measurement in models or in empirical work. But generally, \alert{capital reallocation} is 
	\begin{itemize}
		\item the transfer or sale of capital between productive firms or technologies
		\item sales of property, plant, and equipment (PP\&E) or as acquisitions of entire divisions or firm
	\end{itemize}
\end{frame}


\section{Capital Reallocation: Stylized Facts}
\begin{frame}
	\sectionpage
\end{frame}

\begin{frame}{The basic idea of our paper}
	
	Generally, we can follow \cite{korniotis2013state-level} and test whether province-level business cycle can significantly predict the expected returns.
	\vskip 0.25cm
	This paper can be roughly split into the following three part:
	\begin{enumerate}
		\item Estimate the province-level business cycle. \\(Use direct proxy or use some models to estimate)
		\item Test the predictive power of this indicator. \\(Standard empirical asset pricing testing)
		\item Explain and interpret the predictive power. \\(Find the mechanism)
	\end{enumerate}

\vskip 0.25cm

The first and second part is relatively easy to do, while the third part needs more consideration.
\end{frame}

\section{Capital Reallocation: the Theory}
\begin{frame}
	\sectionpage
\end{frame}

\section{Potential Research}
\begin{frame}
	\sectionpage
\end{frame}

\subsection{Capital Reallocation in China Market}
\begin{frame}
	\begin{centering}
		{\small \begin{beamercolorbox}[center]{part title}
			\usebeamerfont{section title}\insertsubsection\par
		\end{beamercolorbox}}
	\end{centering}
\end{frame}

\subsection{Capital Reallocation and Empirical Asset Pricing}
\begin{frame}
	\begin{centering}
		{\small \begin{beamercolorbox}[center]{part title}
				\usebeamerfont{section title}\insertsubsection\par
		\end{beamercolorbox}}
	\end{centering}
\end{frame}

\section{References}
\begin{frame}
	\sectionpage
\end{frame}


\bibliographystyle{jf}
\bibliography{ref}
\end{CJK*}
\end{document}
